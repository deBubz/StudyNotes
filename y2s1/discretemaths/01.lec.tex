\documentclass[10pt]{beamer}

%\usetheme{m}
\usetheme[titleformat=allsmallcaps, numbering=fraction, block=fill, background=light ]{metropolis} %regular, smallcaps, allsmallcaps, allcaps 
%\setbeamercolor{frametitle}{ fg=mLightBrown , bg= gray}

\useoutertheme{metropolis}
\useinnertheme{metropolis}
\usefonttheme{metropolis}
\usecolortheme{seahorse} % or your preferred color theme

\renewcommand{\geq}{\geqslant} \renewcommand{\leq}{\leqslant} \renewcommand{\ge}{\geqslant} \renewcommand{\le}{\leqslant}


\usepackage{multicol}

\usepackage{xcolor}
\newcommand{\red}[1]{{\color{red!} #1}}
\newcommand{\green}[1]{{\color{green!} #1}}




\usepackage{booktabs}
\usepackage[scale=2]{ccicons}

\usepackage{pgfplots}
\usepgfplotslibrary{dateplot}

\usepackage{oubraces}


\newtheorem{proposition}[theorem]{Proposition}

%%%MURRAY'S LAZY THINGS
\newcommand{\ra}{\to} %{\longrightarrow}
\newcommand{\la}{\longleftarrow}
\newcommand{\df}[2][]{\textbf{\color{black}\sffamily #2}}

\usepackage[framemethod=tikz]{mdframed} % for boxes around things

\usepackage{soul}
\usepackage[normalem]{ulem}               % to striketrhourhg text
\newcommand\redout{\bgroup\markoverwith
{\textcolor{red}{\rule[0.5ex]{2pt}{0.8pt}}}\ULon}


\usepackage{pgf}
\usepackage{tikz}
\usetikzlibrary{arrows,automata}

\usetikzlibrary{chains,decorations.pathmorphing,positioning,fit}
\usetikzlibrary{decorations.shapes,calc,backgrounds}
\usetikzlibrary{decorations.text,matrix}
\usetikzlibrary{arrows,shapes.geometric,shapes.symbols,scopes}
\usetikzlibrary{decorations.markings}


%%%%%%%%%%%%%%%%%%%%%%%%%%%%%%%%%%%%%%%%%%%%%%%%%%%%%%
%%% Mathematical Abbreviations                     %%%
%%%%%%%%%%%%%%%%%%%%%%%%%%%%%%%%%%%%%%%%%%%%%%%%%%%%%%
%Independence Partial Commutation

%%%%%%%% Free groups 
\newcommand{\Apos}{A_{+}}
\newcommand{\Aneg}{A_{-}}
\newcommand{\Apone}{A_{\pm}}
\newcommand{\Bpos}{B_{+}}
\newcommand{\Bneg}{B_{-}}
\newcommand{\Bpone}{A_{\pm}}
\newcommand{\AF}{A_{\F}}
\newcommand{\AFpos}{A_{\F,+}}
\newcommand{\AFpone}{A_{\F,\pm}}

\newcommand{\FG}[1]{\text{F}({#1})}
\newcommand{\FGA}{\FG \Apos}
%%%%%%%%%%%%%% INIT STUFF
\newcommand{\init}{\mathrm{init}}
\newcommand{\ninit}{n_{\mathrm{init}}}
\newcommand{\muinit}{\mu_{\mathrm{init}}}
\newcommand{\Winit}{W_{\mathrm{init}}}



% environments
\newtheorem{thm}{Theorem}[section]
\newtheorem{lem}[thm]{Lemma}
\newtheorem{cor}[thm]{Corollary}
\newtheorem{prop}[thm]{Proposition}
\theoremstyle{definition}
\newtheorem{defn}[thm]{Definition}
\newtheorem{question}[thm]{Question}
\newtheorem{ex}[thm]{Exercise}
\newtheorem{eg}[thm]{Example}
\newtheorem{axiom}[thm]{Axiom}




%%%%%%%%% ARCS 
\newcommand{\arc}[1]{\overset{#1}\ra}
\newcommand{\longarc}[1]{\overset{#1}{\xrightarrow{\hspace*{1.2cm}}}}
%%%% sets { ... | ... }
\newcommand{\set}[2]{\left\{#1\mathrel{\left|\vphantom{#1}\vphantom{#2}\right.}#2\right\}}

\newcommand{\oneset}[1]{\left\{\mathinner{#1}\right\}}
\newcommand{\os}{\oneset}
\newcommand{\sm}{\setminus}
\newcommand{\es}{\emptyset}
\newcommand{\sse}{\subseteq}
\newcommand{\smallset}[1]{\left\{\mathinner{#1}\right\}}

% 2x2-Matrizen und -Vektoren
\newcommand{\VDmatrix}[4]{\begin{pmatrix}{#1}&{#2}\\{#3}&{#4}\end{pmatrix}}
\newcommand{\vdmatrix}[4]{\left(\begin{smallmatrix}#1 & #2\\ #3 & #4\end{smallmatrix}\right)}
\newcommand{\VDvector}[2]{\begin{pmatrix}{#1}\cr{#2}\end{pmatrix}}
\newcommand{\vdvector}[2]{\left(\begin{smallmatrix}#1\cr{#2}\end{smallmatrix}\right)}
%%%%%%%%%%%% brackets etc

\newcommand{\abs}[1]{\left|\mathinner{#1}\right|}
\newcommand{\Abs}[1]{\left\Vert\mathinner{#1}\right\Vert}
\newcommand{\Absmax}[1]{\left\Vert\mathinner{#1}\right\Vert_{\infty}} %Max norm
\newcommand{\Absone}[1]{\left\Vert\mathinner{#1}\right\Vert_{1}} 
%%%%%%% 1 norm = one norm
\newcommand{\Absmaxp}[1]{\left\Vert\mathinner{#1}\right\Vert_{\infty}'} %Max norm prim

\newcommand{\floor}[1]{\left\lfloor\mathinner{#1} \right\rfloor}
\newcommand{\ceil}[1]{\left\lceil\mathinner{#1} \right\rceil}
\newcommand{\bracket}[1]{\left[\mathinner{#1} \right]}
\newcommand{\dbracket}[1]{\left\llbracket \mathinner{#1} \right\rrbracket}
\newcommand{\parenth}[1]{\left(\mathinner{#1} \right)}
\newcommand{\gen}[1]{\left< \mathinner{#1} \right>}
\newcommand{\Gen}[2]{\left< \mathinner{#1} \mid \mathinner{#2}\right>}
\newcommand{\ggen}[1]{\left<\left< \mathinner{#1} \right>\right>}

\newcommand{\scalp}[2]{\langle {#1}\,,\,{#2}  \rangle}
\newcommand{\scalb}[2]{\langle  {#1}\,,\,{#2}  \rangle_{\! \bet}}

\newcommand{\rdeg}[1]{\mathop{\mathrm{red}\mbox{-}\mathrm{deg}}(#1)}
\newcommand{\lcm}[1]{\mathop{\mathrm{lcm}}(#1)}
%%% numbers
\newcommand{\N}{\ensuremath{\mathbb{N}}}
\newcommand{\Z}{\ensuremath{\mathbb{Z}}}
\newcommand{\Q}{\ensuremath{\mathbb{Q}}}
\newcommand{\R}{\ensuremath{\mathbb{R}}}



%%%%%%%%% COMPLEXITY
\newcommand{\PSPACE}{\ensuremath{\mathsf{PSPACE}}}
\newcommand{\TC}{\ensuremath{\mathsf{TC}^0}}
\newcommand{\NC}{\ensuremath{\mathsf{NC}}}
\newcommand{\NP}{\ensuremath{\mathsf{NP}}}
\renewcommand{\P}{\ensuremath{\mathsf{P}}}
\newcommand{\NSPACE}{\ensuremath{\mathsf{NSPACE}}}
\newcommand{\DSPACE}{\ensuremath{\mathsf{DSPACE}}}
\newcommand{\NTIME}{\ensuremath{\mathsf{NTIME}}}
\newcommand{\DTIME}{\ensuremath{\mathsf{DTIME}}}

%%%%%%%%% FORMAL LANG
\newcommand{\edt}{EDT0L\xspace}

%%%%%%%%% Boolean matrices
\newcommand{\Bn}{\B^{n\times n}}
\newcommand{\Mn}{{\B}_{2n}}

%%%%%%%%semi-direct
\newcommand{\sdZ}{\ensuremath{\Z[1/2] \rtimes \Z}}

%%%%%%%%%%% Poly ring 
\newcommand{\Zt}{\Z[t]}

%%%%%%%% ALPHABETS
%%%%GREEK 

\renewcommand{\phi}{\varphi}
\newcommand{\eps}{\varepsilon}
\newcommand{\e}{\eps} % Used for evaluations, so we can change VD.

\newcommand{\oo}{\omega}
\newcommand{\alp}{\alpha}
\newcommand{\bet}{\beta}
\newcommand{\gam}{\gamma}
\newcommand{\del}{\delta}
\newcommand{\lam}{\lambda}
\newcommand{\sig}{\sigma}

\newcommand{\sol}[1]{\ensuremath{\sig(#1)}}
\newcommand{\mysolution}{\ensuremath{\sig}}


\newcommand{\Sig}{\Sigma}
\newcommand{\Gam}{\GG}
\newcommand{\Del}{\Delta}

\renewcommand\SS{\Sigma}
\newcommand\DD{\Delta}
\newcommand\GG{\Gamma}
\newcommand\LL{\Lambda}
\newcommand\Lam{\Lambda}
\newcommand\OO{\Omega}

%%%%%%%%%%% SOME SPECIALS
\newcommand\SL{\mathop\mathrm{SL}}
\newcommand\PSL{\mathop\mathrm{PSL}}

%%%%%%%%%%%%%% MACROS 

\newcommand{\Apm}{A_\pm}
\newcommand{\Red}[1]{\mathop{\mathrm{Red}}(#1)}
\newcommand{\mm}{M(B,\theta,\mu)}
\newcommand{\mmprime}{M(B',\theta',\mu')}


%%%% oh and Oh
\newcommand{\Oh}{\mathcal{O}}
\newcommand{\oh}{o} % {\mathcal{o}}

%%%%% tildes and hats 

\newcommand{\wt}[1]{\widetilde{ #1 }}
\newcommand{\wh}[1]{\widehat{ #1 }}

%%%%%%%%%%%%%%% Identity
\newcommand{\id}[1]{\mathrm{id}_{#1}}
%%%%% caligraphical
\newcommand{\cA}{\mathcal{A}}
\newcommand{\cB}{\mathcal{B}}
\newcommand{\cC}{\mathcal{C}}
\newcommand{\cD}{\mathcal{D}}
\newcommand{\cE}{\mathcal{E}}
\newcommand{\cG}{\mathcal{G}}
\newcommand{\cL}{\mathcal{L}}
\newcommand{\cM}{\mathcal{M}}
\newcommand{\cS}{\mathcal{S}}
\newcommand{\cT}{\mathcal{T}}
\newcommand{\cX}{\mathcal{X}}
\newcommand{\cY}{\mathcal{Y}}
\newcommand{\cZ}{\mathcal{Z}}
\newcommand{\cR}{\mathcal{R}}

%%%%%%%%%% SOLUTION
\newcommand{\cSol}{\mathrm{Sol}}


%%%%%%%%%% overline and inverse
\newcommand{\ov}[1]{\overline{#1}}
\newcommand{\inv}[1]{\ov{#1}}
\newcommand{\oi}[1]{{#1}^{-1}}
\newcommand{\mathinvol}{\overline{\,^{\,^{\,}}}}










\title{37181: Week 1:  Logic}


\date{Wednesday 24 July 2019}
\author{A/Prof Murray Elder, UTS}

\begin{document}

%\maketitle






%
\begin{frame}
\titlepage % Print the title page as the first slide
\vspace{5cm}
\vfill
\end{frame}





%%BEGIN CONTENT FROM HERE
%%NOTE: Use XeLaTeX to compile 



\begin{frame}
\frametitle{Plan}



\begin{itemize}
 \item introduction, subject outline
\bigskip \item truth tables
\bigskip \item logical equivalence
\bigskip \item  tautology
\bigskip \item quantified statements
\bigskip \item negation of quantified statements
\bigskip \item SAT and P=?NP
 \end{itemize}


\vspace{0cm}
\vfill
\end{frame}

%------------------------------------------------




\begin{frame}
\frametitle{Logic}

\begin{definition}
A {\em statement} is a sentence that can (theoretically) be assigned a value of {\em true} or {\em false}.\end{definition}

\bigskip\pause

Eg:
\begin{enumerate}
\item[1.] Um, like, whatever\bigskip
\item[2.] All positive integers are prime\bigskip
\item[3.] All lectures are recorded at UTS\bigskip
\item[4.] In the year 4000BC, at this exact location, it was raining on the 5th of March at 10am\bigskip
\item[5.] When will this lecture end?
\end{enumerate}


\vspace{2cm}
\vfill
\end{frame}
%------------------------------------------------





\begin{frame}
\frametitle{Logical connectives}

We can build up more complicated statements out of simpler ones using  
{\em logical connectives} like {\em and} and {\em or}.

\bigskip

Eg: 
\begin{enumerate}
\item[1.] 
Murray is a statistician and Murray  has brown hair.
\item[2.] 
Murray is a statistician or Murray  has brown hair.
\end{enumerate}
\vspace{2cm}
\vfill
\end{frame}
%------------------------------------------------


\begin{frame}
\frametitle{Precise meaning: truth table}

English (or any natural human language) can be imprecise, so instead of using our 
{\em ``intuitition"} we \textbf{define} what  {\em ``and"} and {\em ``or"} and {\em ``not"} mean using {\em truth tables}.

\pause\bigskip
\[\begin{array}{c|c|ccc}
 p & q & p\wedge q\\
 \hline
 1 & 1 & \\
 1 & 0 & \\
 0& 1 & \\
 0 & 0 & \\
\end{array} \hspace{1cm} \begin{array}{c|c|ccc}
 p & q & p\vee q\\
 \hline
 1 & 1 & \\
 1 & 0 & \\
 0& 1 & \\
 0 & 0 & \\
\end{array}  \hspace{1cm}\begin{array}{c|cccc}
 p &  \neg p\\
 \hline
 1 &  \\
 0&   \\
\end{array}\]


\bigskip\pause

Teenager speech is more precise: Eg:
``Maths is awesome --- NOT"  



\vspace{0cm}
\vfill
\end{frame}
%------------------------------------------------



\begin{frame}
\frametitle{Truth tables for compound statements}



We can use truth tables to decide the truth values of more complicated statements, like $\neg p\vee q$:
\[\begin{array}{c|c|ccc}
 p & q & \neg p & \vee & q\\
 \hline
 1 & 1 & & \\
 1 & 0 &  & \\
 0& 1  &  & \\
 0 & 0 &  & \\
 %&&_{(1)} &_{(2)}
\end{array} \hspace{5cm}.\]
%The order in which we compute columns (according to {\em order} of logical operations) is shown by the \footnotesize
%\tiny${(1)},{(2)}$.

\bigskip
\pause Note that this is different to saying $\neg(p\vee q)$, since the truth values are not the same %(do this)


\vspace{-4.8cm}\[.\hspace{5cm}\begin{array}{c|c|ccc}
 p & q & \neg (p & \vee & q)\\
 \hline
 1 & 1 & & \\
 1 & 0 &  & \\
 0& 1  &  & \\
 0 & 0 &  & \\
 %&&_{(1)} &_{(2)}
\end{array} \]
\vspace{2cm}
\vfill
\end{frame}
%------------------------------------------------



%
%
%\begin{frame}
%\frametitle{Truth tables for compound statements}
%
%
%
%We can use truth tables to decide the truth values of more complicated statements, like $\neg p\vee q$:
%\[\begin{array}{c|c|ccc}
% p & q & \neg p & \vee & q\\
% \hline
% 1 & 1 & 0 & 1\\
% 1 & 0 & 0 & 0\\
% 0& 1  & 1 & 1\\
% 0 & 0 & 1 & 1\\
% &&_{(1)} &_{(2)}
%\end{array}\]
%The order in which we compute columns (according to {\em order} of logical operations) is shown by the \footnotesize
%\tiny${(1)},{(2)}$.
%
%
%\vspace{2cm}
%\vfill
%\end{frame}
%%------------------------------------------------
%


\begin{frame}
\frametitle{Your turn}

Complete the truth tables for these statements:


\[\begin{array}{c|c|ccc}
 p & q & \neg &(p  \wedge  q)\\
 \hline
  1 & 1 &  & \\
 1 & 0 &  & \\
 0& 1  &  & \\
 0 & 0 &  & \\
% 5 &&_{(2)} &_{(1)}
\end{array} \hspace{1cm} 
\begin{array}{c|c|cccc}
 p & q & \neg p &   \vee  & \neg q\\
 \hline
  1 & 1 &  & \\
 1 & 0 &  & \\
 0& 1  &  & \\
 0 & 0 &  & \\
 %5 &&_{(2)} &_{(1)}
\end{array}\]


%\pause $p=$ it is raining  $q=$ today is Tuesday

\vspace{2cm}
\vfill
\end{frame}
%------------------------------------------------


\begin{frame}
\frametitle{Logically equivalent}

When two (compound) statements have the same truth values we say they are {\em logically equivalent}. 
\pause \bigskip

Eg: 
$$\begin{array}{c|c|c|cc}
 p & q & \ \ \  p\vee \neg q  \ \ \  &\ \ \  \neg ( q \wedge \neg p)  \ \  \  \\
 \hline
 1 & 1 &  \\
 1 & 0 &  \\
 0& 1  &  \\
 0 & 0 &  \\
\end{array}$$

\vspace{2cm}
\vfill
\end{frame}
%------------------------------------------------

\begin{frame}
\frametitle{Implies}



In mathematics and logic we have a very specific meaning for ``$p$ {\em implies} $q$'', or ``{\em if} $p$ {\em then} $q$'', notation $p\ra q$.

\bigskip
We define it using the following table:

$$\begin{array}{c|c|ccc}
 p & q &  p\ra q\\
 \hline
 1 & 1 &  1\\
 1 & 0 &  0\\
 0& 1  &  1\\
 0 & 0 &  1\\
\end{array}$$

\bigskip\pause 
You may think that in English, ``{\em if it is raining then I get wet}'' means that the rain {\em caused} me to get wet. But in mathematics {\em if-then} has the meaning defined above: if ``{\em I am wet}" is true and ``{\em it is raining}" is false, the implication is still true.  (I could be  at a swimming pool).

\vspace{2cm}
\vfill
\end{frame}
%------------------------------------------------



\begin{frame}
\frametitle{Your turn}

Show that $p\ra q$ is  logically equivalent to $\neg p\vee q$.
%
%\pause \bigskip
%
%Eg: 
% \[\begin{array}{c|c|c|cc}
% p & q &  &\\ 
%  \hline
%   1 & 1 &   \\
%1 & 0 &   \\
%0& 1  &   \\
%0 & 0 &   \\
% 1 & 1 &   \\
%1 & 0 &   \\
%0& 1  &   \\
%0 & 0 &   \\
%% 5 &&_{(2)} &_{(1)}
%\end{array}\]

\vspace{4cm}
\vfill
\end{frame}
%------------------------------------------------


\begin{frame}
\frametitle{Tautology}



A statement that is true for all truth value assignments is called a {\em tautology}. 

\pause \bigskip

Eg:    
 \[\begin{array}{c|c|ccc}
 p & q & ((p\ra q)\wedge p) \ra q\\
 \hline
    1 & 1 &   \\
   1 & 0 &   \\
   0& 1  &   \\
   0 & 0 &   \\
\end{array}\]

\vspace{2cm}
\vfill
\end{frame}
%--------------------------------------





\begin{frame}
\frametitle{Tautology}




Eg:    
 \[\begin{array}{c|c|c|cc}
 p & q & r& \left[(p\ra q)\wedge (q\ra r)\right] \ra (p\ra r)\\
 \hline
  1 &  1 & 1 &   \\
   1 &1 & 0 &   \\
   1 &0& 1  &   \\
  1 & 0 & 0 &   \\
 0&  1 & 1 &   \\
 0& 1 & 0 &   \\
  0&0& 1  &   \\
  0&0 & 0 &   \\
% 5 &&_{(2)} &_{(1)}
\end{array}\]

\vspace{2cm}
\vfill
\end{frame}
%--------------------------------------



\begin{frame}
\frametitle{Your turn}

Decide which of these are tautologies:

\begin{enumerate}

\item[1.] $((p\ra q)\wedge \neg q) \ra \neg p$
\item[2.] $(p\ra q)\leftrightarrow ( q\ra p)$
\item[3.] $(p\ra q)\leftrightarrow (\neg q\ra \neg p)$
\end{enumerate}
\vspace{4cm}
\vfill
\end{frame}
%--------------------------------------





\begin{frame}
\frametitle{Another way to write tautologies}


In Humanities/Law you might see tautological statements written in this form. Some rules have names.
\bigskip

 \[\begin{array}{lll}
 p\ra q\\
 p\\
 \hline
 q
\end{array}\]
(Modus ponens)

\bigskip\pause
 \[\begin{array}{lll}
 p\ra q\\
\neg q\\
 \hline
 \neg p
\end{array}\]
(Modus tollens)



\vspace{2cm}
\vfill
\end{frame}
%--------------------------------------




\begin{frame}
\frametitle{From wikipedia: }




\begin{tabular}{ll} 
If I am an axe murderer, then I can use an axe.\\
I cannot use an axe.\\
Therefore, I am not an axe murderer.\end{tabular}

\bigskip

Which style of argument is this? (Write it in symbols).

\vspace{2cm}
\vfill
\end{frame}
%--------------------------------------




\plain{pause}



\begin{frame}
\frametitle{Contradiction: preview }
Let $F$ be a statement that is always false (has truth table 0, for example, $F=q\wedge \neg q$). 

\bigskip
Then the statement  $$(\neg p\ra F)\ra p$$ is a tautology. Check it:

\[
\begin{array}{c|c|ccc}
 p & F &   \ \ \ \ \ \ \ \ \ \ \ \ \ \ \ \ \ \ \\
 \hline
 1 & &  \\
 0&   & \\
\end{array}\]



\vspace{5mm}
\pause 
It says, if not $p$ implies something that is false, then it must be $p$ (is true). 
This  argument  form is known as {\em proof by contradiction}. We will study this more when we start {\em proofs}



\vspace{2cm}
\vfill
\end{frame}
%--------------------------------------




\begin{frame}
\frametitle{Variables }

Statements can  contain {\em variables}.

\bigskip
Eg:\begin{itemize}\item 
 $P(x)$: ``the number $x$ is greater than or equal to 3'' \item 
$Q(x)$: ``$x$ lives in Queensland"\end{itemize}

\bigskip\bigskip\pause 
The {\em universe of discourse} is the set of objects over which the statement could be defined.

\bigskip
\begin{itemize}\item for $P(x)$ the universe of discourse could be $\R$ or $\Z$ or $\N$ (we would need to be told)

\bigskip \item for $Q(x)$ the universe might be all people, or all students at QUT.
\end{itemize}

\vspace{2cm}
\vfill
\end{frame}
%--------------------------------------





\begin{frame}
\frametitle{Quantifiers}



We have the symbols $\forall=$``for all''  and $\exists=$``there exists".



\bigskip
Eg: Let the universe of discourse be $\Z$. \begin{itemize}\item 
$\forall x, x^2>x$ reads as ``for all integers $x$, $x^2$ is greater than $x$''

\bigskip

Is this true?
\pause \bigskip

\item 
$\exists x, x^2\leq x$ reads as ``there exists (there is) some  integer $x$ whose square is smaller than or equal to itself''
\bigskip

Is this true?

\end{itemize}
\vspace{2cm}
\vfill
\end{frame}
%--------------------------------------




\begin{frame}
\frametitle{Quantifiers}



Rather than say `` Let the universe of discourse be''
we often hide this (make it {\em implicit}), or write 

\bigskip
\begin{itemize}\item 
$\forall x\in \Z, x^2>x$ 
\end{itemize}
\vspace{2cm}
\vfill
\end{frame}




\begin{frame}
\frametitle{Practice}

Let $B(x)$ be the statement  ``$x$ lives in Bondi". 


Let $C(x)$ be the statement ``$x$ lives in Cabramatta''.
\bigskip\pause

Write these in symbols:

\bigskip
\begin{itemize}\item 
``All UTS students live in Bondi" 
\item \bigskip\bigskip
``All UTS students either live in Cabramatta or do not live in Bondi''
\end{itemize}
\vspace{2cm}
\vfill
\end{frame}
%--------------------------



\begin{frame}
\frametitle{Negation of quantified statements}


Can you work out, {\em intuitively} what the meaning of 

\begin{itemize}\item 
 $\neg\left(\forall x, B(x)\right)$ 
\end{itemize}

is?
\vspace{2cm}
\vfill
\end{frame}
%--------------------------------------


\begin{frame}
\frametitle{Negation of quantified statements}
Formally, to negate a quantified statement you switch $\forall$ and $\exists$ at the front, then negate the proposition.

$$\neg\left(\forall x P(x)\right) = \exists x \neg P(x)$$

\bigskip

$$\neg\left(\exists x P(x)\right) = \forall x \neg P(x)$$

\vspace{2cm}
\vfill
\end{frame}
%--------------------------------------


\begin{frame}
\frametitle{Practice}

Check the course notes, and Week 1 homework sheet, to practice turning English sentences into symbolic statements, and backwards, and negating them.

\vspace{2cm}
\vfill
\end{frame}
%--------------------------------------

\begin{frame}
\frametitle{SAT}


$3$-SAT is the following problem: on input an expression of the form 
$$(x_1\vee y_1\vee z_1)\wedge(x_2\vee y_2\vee z_2)\wedge \dots (x_n\vee y_n\vee z_n)$$
where $x_i,y_i,z_i$ are propositions $p$ or $\neg p$, answer  yes or no: there is some assignment of truth values to the variables which makes the whole statement true.


\bigskip
For example
$$(p\vee q\vee \neg r)\wedge(\neg p\vee q\vee r)\wedge (p\vee \neg q\vee  r)$$

\pause

If I tell you a particular truth assignment, like $p=0, q=1,r=0$ etc, you can easily compute (in a number of steps polynomial in $n$) the truth value of the statement.


\vspace{2cm}
\vfill
\end{frame}
%--------------------------------------

\begin{frame}
\frametitle{SAT}


 If an instance of a solution can be {\em verified} in polynomial time (number of steps), we say a problem is in NP.
 
\bigskip
If a solution can be found in polynomial time (number of steps), we say the problem is in $P$. 

\bigskip\pause
No-one knows if you can always find a truth assignment, or show there is none, making a general $3$-SAT expression true, in polynomially many steps. If you can, you will get \$1M % from the Clay Institute.

\bigskip
$3$-SAT is an important problem, even though it may seem abstract and useless, because Cook and Levin showed that every other candidate to solve the P=NP problem is related to this one. \footnote{hashtag NP-complete}%More details see for example [Sipser].


\vspace{1cm}
\vfill
\end{frame}
%--------------------------------------




\begin{frame}
\frametitle{Coming up}


In your workshop tomorrow/Friday/Monday, lots of practice to fully understand the content presented today. 
\bigskip

After the workshop and before the next lecture (eg: this weekend, or make some time Mon-Tue)
%Before then, try to go through the (blank) slides again and fill them in for yourself, and 
do the homework sheet to consolidate your learning, and be ready for the quiz.

\bigskip
Next lecture: 
\begin{itemize}
 \item proof methods: direct, contrapositive, contradiction

 \end{itemize}




\vspace{2cm}
\vfill
\end{frame}
%------------------------------------------------


\end{document}



